\documentclass[a4paper,11pt]{article}
\usepackage[utf8]{inputenc}
\usepackage[swedish]{babel}
\usepackage{url}
\usepackage{hyperref}
\usepackage{graphicx}
\usepackage[export]{adjustbox}
\usepackage{tocloft}

\hypersetup{
    colorlinks=true,
    allcolors=black
}

\renewcommand{\thesection}{\S\arabic{section}}
\settowidth\cftsecnumwidth{\S 8.8 }

\title{RoyalRoppers Stadgar}
\date{\today}

\setlength{\parindent}{0pt} %Doesn't work!!!

\begin{document}

\maketitle
\newpage

\tableofcontents
\newpage

\section{Namn}
% föreningens namn (association's name)
Föreningens fullständiga namn är RoyalRoppers, nedan benämnd föreningen.

\section{Syfte}
% en utförlig beskrivning av föreningens ideella ändamål (a detailed description of the association's non-profit purpose)
Föreningens syfte är:

\begin{itemize}
    \item att delta i Capture The Flag (CTF) tävlingar,
    \item att vara en samlingsplats för IT-säkerhetsintresserade,
    \item att främja medlemmarnas utbildning och fortbildning inom IT-säkerhet genom utbyte av information,
    \item att vara en brygga mellan sina medlemmar och industrin.
\end{itemize}

\section{Beslutande organ}
Föreningens beslutande organ är ordinarie årsmöte, extrainsatt årsmöte och styrelsen.
Föreningens högsta beslutande organ är årsmötet.

\section{Säte}
% sätesort, det vill säga den ort där styrelsen finns (the location of the seat, i.e. the place where the Board is located)
Föreningen har sitt säte i Stockholm, Sverige.

\section{Verksamhets- och Räkenskapsår}
Föreningens verksamhets- och räkenskapsår är brutet och går från den första november till den sista oktober nästkommande år.


\section{Stadgar}
% regler för ändring av stadgarna (rules for amending the statutes)
\subsection{Ändringar}
För ändring av dessa stadgar krävs beslut på två på varandra följande årsmöten, med minst en månads mellanrum, varav ett ska vara ett ordinarie årsmöte, med minst 2/3 av antalet avgivna röster. Förslag till ändring av stadgarna får skriftligen avges av såväl medlem som styrelse.

\subsection{Tolkning}
Uppstår tvekan om tolkningen av dessa stadgar, eller om fall förekommer som inte är förutsedda i stadgarna, hänskjuts frågan till nästkommande årsmöte. I brådskande fall får frågan avgöras av revisorn. 


\section{Medlemskap}
% regler för medlemskap och uteslutning (rules for membership and exclusion)

\subsection{Medlemmar}

Medlem är den som har lämnat in en av föreningen godkänd medlemsansökan. 
\paragraph{}
Förening är öppen för alla fysiska personer som delar föreningens intressen. Medlem skall följa föreningens stadgar och föreskrifter. Medlemskapet gäller per verksamhetsår. 
\paragraph{}
Medlem som allvarligt skadar föreningen eller motverkar dess syfte kan uteslutas av årsmötet genom beslut med två tredjedels majoritet. I brådskande fall kan styrelsen besluta om interimistisk uteslutning. Sådant beslut ska alltid prövas på nästföljande årsmöte. Ingen får delta i omröstning angående egen uteslutning. Medlem som uteslutits kan efter att minst 6 månader förflutit beviljas återinträde i föreningen efter beslut av styrelsen eller årsmötet.
\paragraph{}
Endast medlemmar kan inneha poster i föreningen.
% Medlemsuppgifter
% - Namn Efternamn
% - Discord
%
% - Är du student?
% - Vart studerar du?
% - Vad studerar du?
% - Vilket år?

\subsection{Behandling av personuppgifter}
Medlem godkänner genom sitt medlemskap att föreningen får behandla personuppgifter i syfte att bedriva ändamålsenlig verksamhet i enlighet med Dataskyddsförordningen, GDPR.


\subsection{Avgift}
% vad som gäller för medlemsavgifter (what applies to membership fees)
Föreningen skall ej ta ut en medlemsavgift. I samband med vissa tävlingar och evenemang kan medlemmar behöva bekosta vissa omkostnader själva.

\subsection{Utträde}
En medlem som önskar utträda ur föreningen, anmäler detta skriftligen till styrelsen varpå denne ej längre anses vara en medlem. Även den medlem som ej förnyar sitt medlemskap i samband med ett nytt verksamhetsår bör anses ha utträtt.

\section{Årsmöte}

% regler för rösträtt och beslutsfattande (information on decision-making bodies)
% regler för kallelse till möten (rules for convening meetings)
% när årsmöte ska hållas (when the annual meeting is to be held)
% vilka frågor som ska behandlas på ett årsmöte (the issues to be dealt with at an annual meeting)
% under vilka omständigheter extra föreningsmöte ska hållas. (the circumstances in which an extraordinary meeting of the Association shall be held.)
% hur kallelsen till föreningsstämman ska ske (how to convene the general meeting)

\subsection{Tidpunkt och kallelse}
\label{sec:arsmote:kallelse}
Årsmötet ska ske någon dag mellan 20 november och 23 december varje år. Kallelse ska utgå per mejl till samtliga medlemmar minst 14 dagar innan årsmötets inträffande. En kompleterande kallelse innehållande samtliga handlingar skall även utgå per mejl senast 5 dagar innan årsmötets inträffande.

\subsection{Beslutsmässighet}
Årsmötet är beslutsmässigt när mötet är behörigt utlyst enligt \ref{sec:arsmote:kallelse}. För beslutsmässighet krävs även att det minsta av 10 medlemmar eller 10\% av föreningens medlemmar är närvarande. Endast motioner som funnits med i handlingarna får behandlas på årsmötet.

\subsection{Motioner}
Samtliga medlemmar får inkomma med motioner inför årsmötet till och med 7 dagar innan årsmötets inträffande.


\subsection{Ärenden vid årsmötet}
Vid årsmötet skall följande behandlas och protokollföras:

\begin{enumerate}
    \item Mötets högtidliga öppnande
    \item Eventuella adjungeringar.
    \item Fastställande av röstlängd för mötet.
    \item Val av ordförande och sekreterare för mötet.
    \item Val av protokolljusterare och rösträknare.
    \item Fråga om mötet har utlysts på rätt sätt.
    \item Fastställande av dagordning.
    \item Styrelsens verksamhetsberättelse för det senaste verksamhetsåret.
    \item Styrelsens förvaltningsberättelse för det senaste räkenskapsåret.
    \item Revisorernas berättelse över styrelsens förvaltning under det senaste verksamhetsåret.
    \item Fråga om ansvarsfrihet för styrelsen för den tid revisionen avser.
    \item Fastställande av verksamhetsplan samt behandling av budget för det kommande verksamhetsåret.
    \item Behandling av propositioner och motioner.
    \item Val av styrelse
    \item Val av revisor
    \item Val av valberedning
    \item Övriga frågor
    \item Mötets högtidliga avslutande
\end{enumerate}

\subsection{Protokoll}
Samtliga beslut tagna av årsmötet samt eventuella reservationer skall protokollföras. Protokollet justeras av de valda protokolljusterarna.

\subsection{Beslut och rösträtt}
Samtliga medlemmar har närvaro-, yttrande-, yrkande- och rösträtt vid mötet. Eventuella icke-medlemmar kan, vid beslut, adjungeras in med närvaro och yttranderätt, men inte rösträtt.
\paragraph{}
Samtliga beslut fattas genom acklamation. Varje deltagare med rösträtt har rätt att begära sluten votering. Om inte annat beslutats gäller enkel majoritet. Beslut där mer än enkel majoritet krävs skall fattas genom sluten votering.

\subsection{Val}
Årsmötet väljer nästa års styrelse, revisor och valberedning.

\subsection{Extrainsatt årsmöte}
Styrelsen kan kalla medlemmarna till extra årsmöte. Styrelsen är skyldig att kalla till extra årsmöte när en revisor eller minst 10 \% av föreningens röstberättigade medlemmar begär det. Sådan framställning skall göras skriftligen och innehålla skälen för begäran. När styrelsen mottagit en begäran om extra årsmöte skall styrelsen inom 14 dagar utlysa sådant möte att hållas inom två månader från erhållen begäran. Vid extra årsmöte får endast det som föranlett mötet upptas till behandling. I övrigt skall ett extrainsatt årsmöte följa samma föresrkifter som ordinarie årsmöte.


\section{Styrelsen}
% bestämmelser om styrelse, dess sammansättning (antal ledamöter och suppleanter), hur den väljs, beslutsfattande (provisions on the board, its composition (number of members and alternates), how it is elected, decision-making)

\subsection{Sammansättning}

Styrelsen består av 8 ledamöter. Däribland skall följande poster tilldelas:

\begin{itemize}
    \item Ordförande
    \item Vice ordförande
    \item Kassör
    \item Sekreterare
\end{itemize}

Ordförande och kassör är firmatecknare och har attesträtt. Styrelseledamöter får inte vara revisor eller valberedning.

\subsection{Uppgifter}
När årsmöte inte är samlat är styrelsen föreningens beslutande organ och ansvarar för föreningens angelägenheter. Styrelsen skall - inom ramen för dessa stadgar - svara för föreningens verksamhet enligt
fastställda planer samt tillvarata medlemmarnas intressen.
Det åligger styrelsen särskilt att:
\begin{itemize}
    \item tillse att för föreningen gällande lagar och bindande regler iakttas,
    \item verkställa av årsmötet fattade beslut,
    \item planera, leda och fördela arbetet inom föreningen,
    \item ansvara för och förvalta föreningens medel,
    \item tillhandahålla alla av revisorn begärda handlingar,
    \item förbereda årsmöte.
\end{itemize}

\subsection{Mandatperiod}
Mandatperioden för styrelseledamöter är ett verksamhetsår.

\subsection{Utträde}
Önskar en styrelseledamot avträda sin post ska detta anmälas på nästkommande styrelsemöte.

\section{Styrelsemöten}
\subsection{Mötets förfarande}
Styrelsen sammanträder efter kallelse av ordföranden eller vice ordförande, eller då minst halva antalet ledamöter har begärt det. Styrelsen är beslutsmässig när samtliga ledamöter kallats och då minst hälften av de valda ledamöterna är närvarande, rundat uppåt. För alla beslut krävs majoritetsröst. Vid lika röstetal har ordföranden utslagsröst. Röstning får inte ske genom ombud.

\subsection{Per capsulam}
I brådskande fall får ordföranden besluta att ärende skall avgöras genom skriftlig omröstning. Sådana beslut skall prövas vid nästkommande styrelsemöte.

\subsection{Protokoll}
Vid sammanträde skall protokoll föras. Protokoll skall justeras av mötesordföranden och mötessekreterare.

\section{Valberedning}
Valberedningen består av en person. Valberedningen har ansvar för att på årsmötet lägga fram ett förslag på styrelse, revisorer samt nästa års valberedning. Förslaget ska även innefatta en fördelning av posterna ordförande, vice ordförande, sekreterare och kassör. Valberedningens mandatperiod är ett verksamhetsår. Valberedningen får inte sitta i styrelsen eller vara revisor.

\section{Revisor}
För granskning av föreningens räkenskaper och styrelsens förvaltning ska utses 1 revisor. Revisorern utses av årsmötet och ska vara oberoende av dem de har att granska. Revisorn får inte sitta i styrelsen eller vara valberedning.
\paragraph{}
Revisorn har rätt att fortlöpande ta del av föreningens räkenskaper, årsmötes- och styrelseprotokoll och
övriga handlingar. Föreningens räkenskaper skall vara revisorn tillhanda senast 20 dagar före ordinarie årsmöte. Revisorern skall granska styrelsens förvaltning och
räkenskaper för det senaste verksamhets- och räkenskapsåret samt till styrelsen överlämna
revisionsberättelse senast 7 dagar före årsmötet. 



\section{Upplösning}
% regler för upplösning med angivande av vad som ska ske med kvarvarande tillgångar. (rules for dissolution, specifying what is to be done with the remaining assets)

Föreningen kan endast upplösas om två på varandra följande ordinarie årsmöten så beslutar med två tredjedelars majoritet. Om minst 10 medlemmar motsätter sig upplösning, kan föreningen inte upplösas. I kallelse till årsmötet skall anges om fråga om upplösning har väckts.
\paragraph{}
Vid upplösning ska föreningens skulder betalas, varefter eventuella återstående tillgångar ska fördelas av årsmötet, på sådant sätt att föreningens syfte främjas. 

% hur beslut ska fattas i föreningen. (how decisions are made in the association)
% regler för hur verksamheten ska bedrivas (rules for the conduct of business)
% uppgift om beslutande organ (information on decision-making bodies)

\end{document}
